%%%%%%%%%%%%%%%%%%%%%%%%%%%%%%%%%%%%%%%%%%%%%%%%%%%%%%%%%%%%%%%
%
% Welcome to Overleaf --- just edit your LaTeX on the left,
% and we'll compile it for you on the right. If you give
% someone the link to this page, they can edit at the same
% time. See the help menu above for more info. Enjoy!
%
%%%%%%%%%%%%%%%%%%%%%%%%%%%%%%%%%%%%%%%%%%%%%%%%%%%%%%%%%%%%%%%
%\title{Math 453 HW 1}
\documentclass[addpoints]{exam}

\usepackage{amsmath,enumitem,wrapfig,amsfonts}
\usepackage{physics}
\usepackage{tikz}

\newcommand{\StudentName}{Sabarno Saha - 22MS037}
\newcommand{\AssignmentName}{Assignement 3}
\newcommand{\subjectcode}{PH2101}

\pagestyle{headandfoot}
\runningheadrule
\runningheader{\subjectcode}{\StudentName}{\AssignmentName}
\firstpagefooter{}{}{\thepage}
\runningfooter{}{}{\thepage}

\printanswers


\begin{document}


\par\textbf{IISER Kolkata} \hfill \textbf{\AssignmentName}
\vspace{3pt}
\hrule
\vspace{3pt}
\begin{center}
        \LARGE{\textbf{\subjectcode : Waves and Optics}}
\end{center}
\vspace{3pt}

\hrule
\vspace{4pt}
\textbf{Sabarno Saha}, \textbf{22MS037}\hfill \today

\vspace{20pt}

\bigskip

\begin{questions}

\question \textbf{ Question 1}\\
Consider a dispersion relation $\omega = vk$, where the symbols have their usual meanings. We
construct a wave packet by choosing sinusoids having a form $e^{i(kx-\omega t)}$ from $-k_0/4$
to $k_0/4$, with uniform amplitude for any k.\\
(a) Calculate the shape for the wave packet.

\begin{solution}\\
 
\end{solution}

(b) What is the group velocity of this packet?
\begin{solution}\\
 
\end{solution}

(c)[Optional] If possible simulate(calculate and animate) and upload how the packet moves with 
time.
\begin{solution}\\
 
\end{solution}

\question \textbf{ Question 2}\\
Consider a damped harmonic oscillator whose equation of motion is given by,
\begin{align*}
    \ddot{x}+\alpha\dot{x}+\omega_0^2x = 0
\end{align*}
where symbols carry their usual meanings.\\ 
(a) Find the solution of the above equation, for given initial conditions $x(0) = x_0$ and 
$\dot{x}(0) = v_0$.
\begin{solution}\\
 
\end{solution}


(b) To find the solution for the critically damped case, check the solution at the limit of the 
vanishing resonance frequency.
\begin{solution}\\
 
\end{solution}

\question \textbf{ Question 3}\\
Consider the following Wave equation:
\begin{align*}
    \pdv[2]{y}{t} = v^2\pdv[2]{y}{x}
\end{align*}
where, $v$ is a constant speed. Check whether this equation is Lorentz invariant. What if
$v=c$, where c is the speed of light in vacuum.
\begin{solution}\\
 
\end{solution}

\question \textbf{ Question 4}\\
Consider a continuous string of length  $L$ whose one end is fixed and the other end is free to
move (it slide on frictionless rods that pass through massless rings at the end of the string).\\ 
(a)Construct the equation of motion (you start with a beaded string and take the continuum
limit)
\begin{solution}\\
 
\end{solution}

(b)Given that the free end is always at the antinode position, find the possible wavelengths.
\begin{solution}\\
 
\end{solution}

\end{questions}
\end{document}
