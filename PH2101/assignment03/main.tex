%%%%%%%%%%%%%%%%%%%%%%%%%%%%%%%%%%%%%%%%%%%%%%%%%%%%%%%%%%%%%%%
%
% Welcome to Overleaf --- just edit your LaTeX on the left,
% and we'll compile it for you on the right. If you give
% someone the link to this page, they can edit at the same
% time. See the help menu above for more info. Enjoy!
%
%%%%%%%%%%%%%%%%%%%%%%%%%%%%%%%%%%%%%%%%%%%%%%%%%%%%%%%%%%%%%%%
%\title{Math 453 HW 1}
\documentclass[addpoints]{exam}

\usepackage{amsmath,enumitem,wrapfig,amsfonts}
\usepackage{physics}
\usepackage{tikz,cancel}

\newcommand{\StudentName}{Sabarno Saha - 22MS037}
\newcommand{\AssignmentName}{Assignement 3}
\newcommand{\subjectcode}{PH2101}

\pagestyle{headandfoot}
\runningheadrule
\runningheader{\subjectcode}{\StudentName}{\AssignmentName}
\firstpagefooter{}{}{\thepage}
\runningfooter{}{}{\thepage}

\printanswers


\begin{document}


\par\textbf{IISER Kolkata} \hfill \textbf{\AssignmentName}
\vspace{3pt}
\hrule
\vspace{3pt}
\begin{center}
        \LARGE{\textbf{\subjectcode : Waves and Optics}}
\end{center}
\vspace{3pt}

\hrule
\vspace{4pt}
\textbf{Sabarno Saha}, \textbf{22MS037}\hfill \today

\vspace{20pt}

\bigskip

\begin{questions}

\question \textbf{ Question 1}\\
Consider a dispersion relation $\omega = vk$, where the symbols have their usual meanings. We
construct a wave packet by choosing sinusoids having a form $Ae^{i(kx-\omega t)}$ from $-k_0/4$
to $k_0/4$, with uniform amplitude for any k.\\
(a) Calculate the shape for the wave packet.

\begin{solution}\\
    The wave packet has a form $\psi(x,t)=e^{i(kx-\omega t)}$. So we intgrate from the possible wavenumbers.
    \begin{align*}
        \text{Wave form shape} =& \int \psi(x-vt) dk\\ 
        =& A\int_{-\frac{k_0}{4}}^{\frac{k_0}{4}} e^{i(kx-\omega t)} dk \\ 
        \tag{Using dispersion relation $\omega = vk$}
        =& A\int_{-\frac{k_0}{4}}^{\frac{k_0}{4}} e^{ik(x-vt)}dk \\ 
        =& A\cfrac{-ie^{ik(x-v t)}}{x-vt}\Bigg|_{-\frac{k_0}{4}}^{\frac{k_0}{4}}  dk \\ 
        =& -iA\cfrac{e^{i\frac{k_0}{4}(x- vt)} - e^{-i\frac{k_0}{4}(x- vt)}}{x-vt}\\ 
        =& A\cfrac{\sin\qty(\frac{k_0}{4}(x-vt))}{x-vt} \\ 
        \Rightarrow\psi(x,t)=& A\cfrac{\sin\qty(\frac{k_0}{4}(x-vt))}{x-vt} \\ 
    \end{align*}
    Thus we have the wave packet of the form $\psi(x,t)= A\cfrac{\sin\qty(\frac{k_0}{4}(x-vt))}{x-vt}$, where
    $A$ is an arbitrary constant which just scales the amplitude of the wave packet.

\end{solution}

(b) What is the group velocity of this packet?
\begin{solution}\\
    We have defined Group velocity to be $v_g = \frac{d\omega}{dk}$ where $\omega$ is the angular frequency and $k$ is the angular wave number. We have from the dispersion relation
    $\omega = vk$. Thus $\dv{\omega}{k} = v$ which is the group velocity of the wave packet.\\ 
    \begin{center}
        \fbox{$v_g = v$}
    \end{center}
\end{solution}

(c)[Optional] If possible simulate(calculate and animate) and upload how the packet moves with 
time.
\begin{solution}\\
 
\end{solution}

\question \textbf{ Question 2}\\
Consider a damped harmonic oscillator whose equation of motion is given by,
\begin{align*}
    \ddot{x}+\alpha\dot{x}+\omega_0^2x = 0
\end{align*}
where symbols carry their usual meanings.\\ 
(a) Find the solution of the above equation, for given initial conditions $x(0) = x_0$ and 
$\dot{x}(0) = v_0$.
\begin{solution}\\
    We aim to solve this via the characteristic polynomial. Let us consider $x = e^{rt}$ to be a 
    solution.
    \begin{align*}
        &\ddot{x}+\alpha\dot{x}+\omega_0^2x = 0 \\ 
        \Rightarrow&r^2e^{rt} + \alpha re^{rt} + \omega_0^2e^{rt} = 0\\ 
        \Rightarrow&e^{rt}\qty(r^2 + \alpha r+ \omega_0^2) = 0\\ 
        \tag{Since $e^{rt} \ne 0 ~~\forall ~r\in \mathbb{R}$}
        \Rightarrow& r^2 + \alpha r + \omega_0^2 = 0\\ 
        \Rightarrow& r = \dfrac{-\alpha \pm \sqrt{\alpha^2 - 4\omega_0^2}}{2}
    \end{align*}
    Thus we put in the values of $r$ in the guessed value of $x$ to get the solution as
    \begin{align*}
        x(t) = e^{-\frac{\alpha}{2}t}\qty(Ae^{\cfrac{\sqrt{\alpha^2 - 4\omega_0^2}}{2}t} +Be^{-\cfrac{\sqrt{\alpha^2 - 4\omega_0^2}}{2}t})
    \end{align*}
    Now we come to the IVP,\\ 
    \begin{align*}
        \tag{we have $x(0) = x_0$}
        x(0) &= A+B = x_0 \\ 
        \tag{also we have $\dot{x}(0) = v_0$. Let $\cfrac{\sqrt{\alpha^2 - 4\omega_0^2}}{2} = \gamma$}
        \dot{x}(0) &= \gamma A - \gamma B - \frac{\alpha}{2}(A+B) = v_0 \\
    \end{align*}
    Solving for A and B we have,
    \begin{align*}
        A + B =& x_0 \\ 
        A - B =& \cfrac{v_0}{\gamma} + \frac{\alpha x_0}{2\gamma} \\ 
        A =& \cfrac{x_0 + \cfrac{v_0}{\gamma}+ \cfrac{\alpha x_0}{2\gamma}}{2}\\ 
    B =& \cfrac{x_0 - \cfrac{v_0}{\gamma} - \cfrac{\alpha x_0}{2\gamma}}{2}\\
    \end{align*}
    \begin{center}
        \fbox{$ x(t) = e^{-\frac{\alpha}{2}t}\qty[\cfrac{x_0 + \cfrac{v_0}{\gamma}+\cfrac{\alpha x_0}{2\gamma}}{2}\qty(e^{\gamma t}) +\cfrac{x_0 - \cfrac{v_0}{\gamma}-\cfrac{\alpha x_0}{2\gamma}}{2}\qty(e^{-\gamma t})]$}
    \end{center}
\end{solution}


(b) To find the solution for the critically damped case, check the solution at the limit of the 
vanishing resonance frequency.
\begin{solution}\\
    We have from previous question $ x(t) = e^{-\frac{\alpha}{2}t}\qty[\cfrac{x_0 + \cfrac{v_0}{\gamma}+\cfrac{\alpha x_0}{2\gamma}}{2}\qty(e^{\gamma t}) +\cfrac{x_0 - \cfrac{v_0}{\gamma}-\cfrac{\alpha x_0}{2\gamma}}{2}\qty(e^{-\gamma t})]$
        \begin{align*}
            x(t) &= e^{-\frac{\alpha}{2}t}\qty[x_0 \cosh(\gamma t) + \frac{2v_0 + \alpha x_0}{2\gamma}\sinh(\gamma t)]\\ 
            \tag{Taking limit $\gamma \rightarrow0$}
                 &= e^{-\frac{\alpha}{2}t}\qty[x_0 + \frac{2v_0+\alpha x_0}{2\cancel{\gamma}}(\cancel{\gamma} t)]\\ 
                 &= e^{-\frac{\alpha}{2}t}\qty[x_0 +\qty( \cfrac{\alpha x_0}{2}+ v_0) t)]\\ 
        \end{align*}
        Thus we have the solution for the critically damped case.
        \\ 
        \textbf{Note:} \\ 
        The limit is : \begin{align*}\lim_{x\to 0} \frac{\sinh x}{x} &= \lim_{u\to 0} \frac{\sinh (iu)}{iu}\tag{Let $x=iu$} \\ &= \lim_{u\to 0} \frac{i\sin (u)}{iu} \\ &= \lim_{u\to 0} \frac{\sin (u)}{u} \\ &=1\end{align*}
\end{solution}

\question \textbf{ Question 3}\\
Consider the following Wave equation:
\begin{align*}
    \pdv[2]{y}{t} = v^2\pdv[2]{y}{x}
\end{align*}
where, $v$ is a constant speed. Check whether this equation is Lorentz invariant. What if
$v=c$, where c is the speed of light in vacuum.
\begin{solution}\\
    Let us have two inertial frames S and S'. S' has a velocity $V$ wrt S along the $x$ axis,
    where the origins of the frames coincide at $t=0$. The equation of lorentz trnasforms are,
    \begin{align*}
        \tag{$\gamma =\sqrt{1-\frac{V^2}{c^2}} $}
        x' &= \gamma(x-Vt) \\ 
        t' &= \gamma(t - \frac{V}{c^2}x)\\ 
        y'&=y
    \end{align*}
    Now we transform the wave equation by taking partial derivatives using $x'$ and $t'$.
    \begin{align*}
        \pdv{y}{t} &= \pdv{y}{x'}\pdv{x'}{t} + \pdv{y}{t'}\pdv{t'}{t} \\ 
                   &= -\gamma V\pdv{y}{x'} + \gamma \pdv{y}{t'} \\ 
        \tag{Let $\pdv{y}{t} = \Omega $}
        \pdv[2]{y}{t} &= \pdv{\Omega}{x'}\pdv{x'}{t} + \pdv{\Omega}{t'}\pdv{t'}{t}\\ 
                      &= -\gamma V\pdv{\Omega}{x'} + \gamma \pdv{\Omega}{t'}\\ 
                      \tag{Substituting $y=y'$ and $\Omega$}
                      &=\gamma^2\qty(V^2\frac{\partial^2 y'}{\partial x'^2}+\frac{\partial^2 y'}{\partial t'^2}-2V\pdv[2]{y'}{x'}{t'})
    \end{align*}
    Similarly,
    \begin{align*}
        \pdv{y}{x} &= \pdv{y}{x'}\pdv{x'}{x} + \pdv{y}{t'}\pdv{t'}{x} \\ 
                   &= \gamma \pdv{y}{x'} - \gamma \frac{V^2}{c}\pdv{y}{t'} \\ 
        \tag{Let $\pdv{y}{x} = \Omega $}
        \pdv[2]{y}{x} &= \pdv{\Omega}{x'}\pdv{x'}{x} + \pdv{\Omega}{t'}\pdv{t'}{x} \\
                      &= -\gamma \frac{V^2}{c}\pdv{\Omega}{x'} + \gamma \pdv{\Omega}{t'} \\ 
                      \tag{Substituting $y=y'$ and $\Omega$}
                      &=\gamma^2\qty(\frac{\partial^2 y'}{\partial x'^2}+\frac{V^2}{c^4}\frac{\partial^2 y'}{\partial t'^2}-2\frac{V}{c^2}\pdv[2]{y'}{x'}{t'})
    \end{align*}
    In the S frame we have the equation $v^2\pdv[2]{y}{x} = \pdv[2]{y}{t}$. Equating in S' frame
    we get,
    \begin{align*}
        &\gamma^2\qty(V^2\frac{\partial^2 y'}{\partial x'^2}+\frac{\partial^2 y'}{\partial t'^2}-2V\pdv[2]{y'}{x'}{t'})=v^2\gamma^2\qty(\frac{\partial^2 y'}{\partial x'^2}+\frac{V^2}{c^4}\frac{\partial^2 y'}{\partial t'^2}-2\frac{V}{c^2}\pdv[2]{y'}{x'}{t'})\\
    \Rightarrow& \frac{\partial^2 y'}{\partial t'^2}\qty(1-\frac{v^2V^2}{c^4}) =v^2\frac{\partial^2 y'}{\partial x'^2}\qty(1-\frac{V^2}{v^2})+2V\pdv[2]{y'}{x'}{t'}\qty(1-\frac{v^2}{c^2})
    \end{align*}
    We can see that the equations are not invariant under Lorentz transformations. Now we check if 
    equations are galilean invariant. We can check this just by taking the galilean limit $\frac{V}{c}\rightarrow0$
    We get,
    \begin{align*}
        \frac{\partial^2 y'}{\partial t'^2}=v^2\frac{\partial^2 y'}{\partial x'^2}\qty(1-\frac{V^2}{v^2})+2V\pdv[2]{y'}{x'}{t'}\qty(1-\frac{v^2}{c^2})
    \end{align*}
    We now check if the equation is Lorentz invariant. In fact this equation is one of the reasons that 
    led to the development of the special theory of relativity. Putting $v=c$, we obtain the 
    equation followed by light.
    \begin{align*}
        &\frac{\partial^2 y'}{\partial t'^2}\qty(1-\frac{c^2V^2}{c^4}) =c^2\frac{\partial^2 y'}{\partial x'^2}\qty(1-\frac{V^2}{c^2})+\cancel{2V\pdv[2]{y'}{x'}{t'}\qty(1-\frac{c^2}{c^2})}\\
        \Rightarrow&\frac{\partial^2 y'}{\partial t'^2}\cancel{\qty(1-\frac{V^2}{c^2})} =c^2\frac{\partial^2 y'}{\partial x'^2}\cancel{\qty(1-\frac{V^2}{c^2})}\\
        \Rightarrow&\frac{\partial^2 y'}{\partial t'^2} =c^2\frac{\partial^2 y'}{\partial x'^2}
    \end{align*}
    We thus see that the wave equation is Lorentz Invariant when the wave travels with velocity 
    equal to the speed of light.
\end{solution}

\question \textbf{ Question 4}\\
Consider a continuous string of length  $L$ whose one end is fixed and the other end is free to
move (it slide on frictionless rods that pass through massless rings at the end of the string).\\ 
(a)Construct the equation of motion (you start with a beaded string and take the continuum
limit)
\begin{solution}\\
   Using the notion used in class we take the transverse displacements of the beads to be $y_k$
   where k represents the $k^{th}$ bead.
   \begin{align*}
       &m\ddot{y_n} = -k(y_{n+1}-y_n)-k(y_n-y_{n-1})\\ 
       \Rightarrow&\ddot{y_n} = -\frac{k}{m}(2y_n-y_{n+1}-y_{n-1})\\  
       \tag{Let be $y_n$ be $y(x)$}
       \Rightarrow&\ddot{y} = \frac{k}{m}(y(x+L)-y(x) - y(x) + y(x-L))\\ 
       \tag{dividing and multiplying by $L^2$ on RHS }\\ 
       \Rightarrow&\ddot{y} = \frac{kL^2}{m} \cfrac{\cfrac{y(x+L)-y(x)}{L}-\cfrac{y(x)-y(x-L)}{L}}{L}
        \tag{taking limit $L\rightarrow0$ and let $\frac{kL^2}{m}= v^2$ for some v}\\ 
       \Rightarrow&\pdv[2]{y}{t} = v^2 \pdv[2]{y}{x}
   \end{align*}
\end{solution}

(b)Given that the free end is always at the antinode position, find the possible wavelengths.
\begin{solution}\\
    We solve this by Variable seperation 
    \begin{align*}
        &y(x,t) = y_x(x)y_t(t) \\ 
        \tag{Using the wave equation we get and since it holds for all $x,t$}\\ 
        \tag{where $\omega^2$ is an arbitrary const}
    \Rightarrow& \frac{1}{y_t}\pdv[2]{y_t}{t} = \frac{v^2}{y_x} = -\omega^2
    \end{align*}
    Thus we have a solution of $y(x,t)=y_x(x)y_t(t)$
    \begin{align*}
        y_x(x) &= A\sin\qty(\frac{\omega}{v}x+\phi_x)\\ 
        y_t(t) &=B \sin\qty(\omega t+\phi_t)
    \end{align*}
    Now applying boundary conditions we will get the possible wavelengths. $y_x(0)=0$. Thus 
    $\sin(\phi_x) = 0 $. Thus $\phi_x = n\pi$. Let us choose $\phi_x = 0$. Then 
    $y_x(x) = A\sin\qty(\frac{\omega}{v}x)$. The second boundary condition tell us that there is 
    an antinode at $x=L$, which says that $y_x$ will take maximum value there.
    \begin{align*}
        &\sin\qty(\frac{\omega}{v}L) = 1 \\ 
        \tag{where $k~~ \in ~~\mathbb{N} $}
        \Rightarrow& \frac{\omega}{v}L = (2k-1)\frac{\pi}{2} \\ 
        \tag{we know $\frac{\omega}{v} = \frac{2\pi}{\lambda}$}
        \Rightarrow& \lambda = \frac{4L}{2k-1}
    \end{align*}
    Which are the possible values of $\lambda$.
    () 
\end{solution}

\end{questions}
\end{document}
