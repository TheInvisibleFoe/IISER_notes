\documentclass[a4paper]{article}

\usepackage[utf8]{inputenc}
\usepackage[T1]{fontenc}
\usepackage{textcomp}
\usepackage[dutch]{babel}
\usepackage{amsmath, amssymb}
% \usepackage{preamble}
\usepackage{mlmodern}
\usepackage{transparent}
\newcommand{\incfig}[1]{%
    \def\svgwidth{\columnwidth}
    \import{./figures/}{#1.pdf_tex}
}
\pdfsuppresswarningpagegroup=1
\title{LS2203 Lab Report}
\author{Sabarno Saha}
\begin{document}
\maketitle
\section{Isolation of Plasmid DNA using alkaline lysis}
\subsection{Principle}
The most common method of plasmid extraction is Alkaline Lysis. The principle is using an alkaline buffer 
using a detergent like sodium dodecylsulphate(SDS) and a strong base like sodium hydroxide(NaOH) to break 
cells open. Then the purification of plasmid DNA is done on the basis of differential denaturation of 
chromosomal and plasmid DNA using Alkaline lysis to seperate the two types of DNA.

\subsection{Materials Used}
We are using three solutions named creatively Solution I, II and III. The compositions of the Solutions 
are given below:
\begin{itemize}
    \item \textbf{Solution 1: }\\ 
         50 mM glucose($C_6H_{12}O_6$), 25 mM Tris HCl(pH 8.0), 10 mM EDTA(pH 8.0). This prepared solution 
         is autoclaved for 15 minutes at 10 lb/in2 on a liquid cycle and stored at 4° C.
         \begin{itemize}
           \item Glucose : maintains osmotic pressure in the cells.
           \item Tris HCl: functions as a chemical to keep a stabel pH.
            \item  EDTA : Acts as a chelating agent to chelate $Mg^{2+}$ ions to keep DNAse from damaging 
                plasmid DNA
         \end{itemize}
\end{itemize}


    
\end{document}
