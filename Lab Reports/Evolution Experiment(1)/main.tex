\documentclass{scrartcl}
\usepackage{handout}
\usepackage{pgfplots,pgf-pie, multirow}
\pgfplotsset{compat=1.18}
\usepackage{tikz}

\setlength{\tabcolsep}{10pt}
\setlength{\arrayrulewidth}{0.6pt}
\renewcommand{\arraystretch}{1.175}

\title{Experiment on Evolution}
\author{Sabarno Saha (22MS037)\\Rishab P Hariharan (22MS045)}
\date{}

\begin{document}
\maketitle
\tableofcontents
\newpage

\section{Introduction and Question}
\begin{question*}
   Design a simple experiment that you can use to demonstrate the concept of evolution through
    natural selection to students of classes VI to VIII in a classroom or outside the school.
Remember: 

    \begin{enumerate}
        \item  They are not exposed to the idea of natural selection.
        \item  The experiment should be simple so that anyone can conduct it.
        \item The experiment should be inexpensive.
    \end{enumerate}
\end{question*}
\subsection{A small introduction}
Evolution of large populations can be easily and cheaply simulated. We can introduce multiple types of species, constructed by simple materials like pebbles and coloured paper. Then we could easily introduce predators and see how the population changes.
\section{Experiment}
\subsection{Materials}
\begin{enumerate}
    \item Pebbles
    \item Coloured paper 
    \item A small fan
\end{enumerate}
\subsection{Procedure}
\begin{itemize}
    \item \textbf{Part 1} We get colored paper of different colors, one green, one brown(soil coloured), one grey, and one black.
    \item We pick pebbles and wrap them up in different coloured papers(speciation). Now we cut out pieces of coloured paper as well( to make a different species).

    \item Now we pick our habitats which will be a green grass field( we can make it a small section of a field to make the exercise easier later), a section of a concrete road, a small dirt patch, and a patch of asphalt road.
    \item Now we mix the different colored pebbles and colored paper up and spread them out in different habitats.
    \item Now we get groups of students and assign them to different habitats and make them find out number of items from each habitat.
    \item Something note making the number of colours in a certain habitat equally numbers would help later in data Analysis as we would not have to do rather complicated calculations to properly model data. However it would still work fairly well as a qualitative experiment.
    \item Here the groups of students are the predators finding out species of coloured paper and coloured pebbles which are the prey. 
    \item \textbf{Part 2} Now for the second part of the experiment it could happen that the predators have some kind of a special modification to find out a certain kind of species in a certain environment. 
    \item For the second part we give the groups small fans. What this does is the coloured papers are blown away by the wind from the fan allowing them to find out a larger number of coloured papers than before, but not the coloured pebbles since they wont be affected by the fan.
    \item Now we could demonstrate evolution another way as well.
    \item In the natural world animals evolve so they would survive in a certain habitat. Now what we could do is make the students get different shades of paper so that the number of same coloured species in habitats of the same color decreases continually. Essentially, what it means that the students try to find colours that match the shade of the habitat as closely as possible to help the prey to not get preyed upon.
\end{itemize}

\section{Conclusion}
 We could predict and eventually see that the when the species whose colour matches the colored of the habitat gets found less in that habitat. So hereby we conclude our experiment.
\end{document}
