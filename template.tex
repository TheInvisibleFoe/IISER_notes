%%%%%%%%%%%%%%%%%%%%%%%%%%%%%%%%%%%%%%%%%%%%%%%%%%%%%%%%%%%%%%%
%
% Welcome to Overleaf --- just edit your LaTeX on the left,
% and we'll compile it for you on the right. If you give
% someone the link to this page, they can edit at the same
% time. See the help menu above for more info. Enjoy!
%
%%%%%%%%%%%%%%%%%%%%%%%%%%%%%%%%%%%%%%%%%%%%%%%%%%%%%%%%%%%%%%%
%\title{Math 453 HW 1}
\documentclass[addpoints]{exam}

\usepackage{amsmath,enumitem,wrapfig}
\usepackage{tikz}

\newcommand{\StudentName}{Sabarno Saha - 22MS037}
\newcommand{\AssignmentName}{HW <No>}

\pagestyle{headandfoot}
\runningheadrule
\runningheader{SB1101}{\StudentName}{\AssignmentName}
\firstpagefooter{}{}{\thepage}
\runningfooter{}{}{\thepage}

\printanswers


\begin{document}


\par\textbf{IISER Kolkata} \hfill \textbf{Assignment <No>}
\vspace{3pt}
\hrule
\vspace{3pt}
\begin{center}
        \LARGE{\textbf{<Sub Code> 1101 : <Subject>}}
\end{center}
\vspace{3pt}

\hrule
\vspace{4pt}
\textbf{Sabarno Saha}, \textbf{22MS037}\hfill \today

\vspace{20pt}

\bigskip

\begin{questions}

\question \textbf{ Question 1}

\begin{solution}\\
 
\end{solution}

\question \textbf{ Question 2}

\begin{solution}\\
 
\end{solution}

\question \textbf{ Question 3}

\begin{solution}\\
 
\end{solution}

\question \textbf{ Question 4}

\begin{solution}\\
 
\end{solution}


\question \textbf{ Question 5}

\begin{solution}\\
 
\end{solution}

\question \textbf{ Question 6}

\begin{solution}\\
 
\end{solution}

\question \textbf{ Question 6}

\begin{solution}\\
 
\end{solution}

\question \textbf{ Question 6}

\begin{solution}\\
 
\end{solution}

\question \textbf{ Question 6}

\begin{solution}\\
 
\end{solution}

\question \textbf{ Question 6}

\begin{solution}\\
 
\end{solution}

\question \textbf{ Question 6}

\begin{solution}\\
 
\end{solution}






\end{questions}
\end{document}
